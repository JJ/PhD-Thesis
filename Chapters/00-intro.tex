\myChapter{Introduction}\label{chap:introduction}
\begin{flushright}{\slshape
    .} \\ \medskip
    --- {}
\end{flushright}
\minitoc\mtcskip
\vfill



\section{Goals of this Thesis} %FERGU: esto es un párrafo que describe el objetivo de la tesis.

\lettrine{T}{he} goal of this thesis is to demonstrate that with an adequate data processing of the context information related to a user interacting with a device, an by analysing user's past behaviour, it is possible to adapt and extend an existing set of security rules in order to respond more accurately and faster to threats.

\section{Motivation}
\label{sec:intro:eas}

...

\section{Challenges in BYOD}
\label{sec:intro:challenges}

...

\section{Objectives}                     
\label{sec:intro:motivation}

...

\newcommand{\objectiveparadigm}{}

 \subsection*{Objective 1: \objectiveparadigm}
\label{subsec:intro:obj:problems}

\newcommand{\objectivemethodology}{}

\subsection*{Objective 2: \objectivemethodology} 
\label{subsec:intro:obj:methodology}

\newcommand{\objectiveframework}{}

\subsection*{Objective 3: \objectiveframework}
\label{subsec:intro:obj:fwork}

\newcommand{\objectiveresearch}{} 

\subsection*{Objective 4: \objectiveresearch}
\label{subsec:intro:obj:applications}



\section{Structure of the thesis}
\label{sec:intro:structure}


%\begin{SCfigure}[tb]
%\centering
% \includegraphics[scale =0.3] {gfx/intro/tesispiramide.pdf}
%\caption{Summary of the objectives of this thesis.}
%\label{fig:intro:piramid}
%\end{SCfigure}
