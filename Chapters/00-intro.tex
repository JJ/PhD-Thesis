\myChapter{Introduction}\label{chap:introduction}
\begin{flushright}{\slshape
    .} \\ \medskip
    --- {}
\end{flushright}
\minitoc\mtcskip
\vfill



\section{Goals of this Thesis} %FERGU: esto es un párrafo que describe el objetivo de la tesis.

\lettrine{T}{he} goal of this thesis is to demonstrate that with an adequate data processing of the context information related to a user interacting with a device, an by analysing user's past behaviour, it is possible to adapt and extend an existing set of security rules in order to respond more accurately and faster to threats.

\section{Motivation}
\label{sec:intro:eas}

The fast pace of introduction of new technologies, including the so-called \textit{smart devices}, has led modern computing to undergo several outstanding transitions in a short period of time. Modern computing has moved over time to smaller, more reliable, and faster devices such as smartphones, laptops, and tablets. At the same time, their use in several forms is progressing and among others, has given birth to the form of use known as the ``Bring Your Own Device'' (BYOD) concept or philosophy. Since its first appearance in research \cite{ballagas2004byod} as a way to name the interaction between people's devices and a public display, such as a piece of art or advertising, it has become a very popular practise, integrated into companies \cite{thomson2012byod} and even schools \cite{song2014bring}.

More precisely, in the corporate world the BYOD practice refers to allowing employees to use their personal laptops, smartphones, tablets, and other mobile devices
for work-related tasks, while not necessarily being in the workplace. This has many advantages \cite{singh2012byod}; among them it is worth to mention saving costs --
since the company saves money on high-priced devices that it would normally be required to purchase for their employees --, and increasing flexibility and worker productivity as employees will not be asked to haul around while on the road multiple devices to satisfy both their personal and work needs, having everything they need in one device anytime and anywhere. Other advantages are tied to the increase of worker satisfaction, attracting the best candidates, and the increase of engagement in the workplace and after hours \cite{singh2012byod}. 

\section{Challenges in BYOD}
\label{sec:intro:challenges}


\subsection{Context awareness}
\label{subsec:context}

% Sensors, precision... etc

\subsection{Platform independence}
\label{subsec:platf_ind}

% Different OSs, architectures...

\subsection{Open science}
\label{subsec:opS}

% I know i'm barely copying Pablo but here we can talk about not open datasets also and their problems (privacy).

\section{Objectives}                     
\label{sec:intro:motivation}

...

\newcommand{\objectivescenarios}{To describe the target environments, or use cases, to take into account in order to develop a the metrics, techniques, and problems to solve.} % to describe or to define?

 \subsection*{Objective 1: \objectivescenarios}
\label{subsec:intro:obj:problems}

\newcommand{\objectivemetrics}{To define a set of metrics able to measure the efficiency of a particular set of security rules.}

\subsection*{Objective 2: \objectivemetrics} 
\label{subsec:intro:obj:methodology}

\newcommand{\objectivetechniques}{To choose which Soft Computing and Machine Learning techniques suit best for every defined scenario.}

\subsection*{Objective 3: \objectiveframework}
\label{subsec:intro:obj:fwork}

\newcommand{\objectiveresearch}{} 

\subsection*{Objective 4: \objectiveresearch}
\label{subsec:intro:obj:applications}



\section{Structure of the thesis}
\label{sec:intro:structure}


%\begin{SCfigure}[tb]
%\centering
% \includegraphics[scale =0.3] {gfx/intro/tesispiramide.pdf}
%\caption{Summary of the objectives of this thesis.}
%\label{fig:intro:piramid}
%\end{SCfigure}
